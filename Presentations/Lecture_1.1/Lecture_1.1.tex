
%%% Uncomment for slide version
\documentclass{beamer}
\setbeameroption{hide notes} % Only slides

%%% Uncomment for handout version
%\documentclass[handout]{beamer}
%\setbeameroption{show notes on second screen=right} % Both


\usepackage{listings}
\usepackage{adjustbox} % To incorporate code into Latex
\usepackage{multirow} % To merge multiple rows  in a table
\usepackage{soul} % To put in strikethrough text


\setbeamertemplate{note page}{\pagecolor{white}\insertnote}
\setbeamertemplate{footline}{}
\usetheme[progressbar=frametitle]{moloch}% modern fork of the metropolis theme
\setbeamercolor{background canvas}{bg=white}
\setbeamercolor{progress bar}{use=palette primary,fg=black,bg=black}
\setbeamercolor{note page}{bg=white} 
\setbeamertemplate{date}{}





\setbeamertemplate{page number in head/foot}{}


\addtobeamertemplate{navigation symbols}{}{%
	\usebeamerfont{footline}%
	\usebeamercolor[fg]{footline}%
	\hspace{1em}%
	\insertframenumber/\inserttotalframenumber
}
\setbeamercolor{itemize item}{fg=black}
\setbeamercolor{itemize subitem}{fg=black}
\setbeamercolor{itemize subsubitem}{fg=black}

\newcommand\blfootnote[1]{%
	\begingroup
	\renewcommand\thefootnote{}\footnote{#1}%
	\addtocounter{footnote}{-1}%
	\endgroup
}


%%%%%%%%%%%%%%%%%%
%%%%%%%%%%%%%%%%%%
%%%%%%%%%%%%%%%%%%
%%%%%%%%%%%%%%%%%%
%%%%%%%%%%%%%%%%%%
%%%%%%%%%%%%%%%%%%
%%%%%%%%%%%%%%%%%%




\title{\Huge FRST302: Forest Genetics}
\author{\Large Lecture 1.1: Classical Genetics and its Molecular Mechanisms}
\date{\today}

\begin{document}
	\maketitle

\note{\emph{Remember, everything on the lecture slides and the accompanying notes is potentially examinable!}}
% for the beamer version
%\documentclass{beamer}


%%% Slide 2
	
\begin{frame}
		\frametitle{Outline for Today}
\setbeamertemplate{itemize items}[circle]
\Large{
			\begin{itemize} 
			\item Short history of genetics
			\item Mendel's laws
			\item Chromosomes
		\end{itemize}
	}

\note{
Learning Outcomes
	\emph{		\begin{itemize} 
				\item Basic definitions in genetics
				\item Principles and terms in classical genetics
				\item Molecular mechanisms of classical genetics
				\item Chromosome crossover and its significance
			\end{itemize}
	}
}
\end{frame}

%%% Slide 3
\begin{frame}
\frametitle{History of Genetics}

\Large \textbf{What is genetics?} \par

\bigskip
\pause
\Large \textbf{Genetics is the study of genes}, of variation and heredity across all branches of the tree of life




\end{frame}



%%% Slide 4

\begin{frame}
	
	\Huge \centering \emph{What are the major questions in genetics?}
	\vspace{20pt}
	\includegraphics[keepaspectratio, width  =0.8\textwidth]{img/pollenPlume}
	\note{
		The answer to this question totally depends on your perspective. However, I think that it is more than fair to say that the following questions are at the heart of most biological science:
		\begin{itemize} 
			\item Why is there so much variation among individuals?
			\item How is this variation maintained in populations?
			\item Why do offspring tend to resemble their parents? 
			\par
		\end{itemize}

		\footnote \url{https://www.asthmacenter.com/wp-content/uploads/Pine-Pollen-Plume-e1495119706845.jpg}
	}
\end{frame}

%%% Slide 5


%%% Slide 6
\begin{frame}
	\frametitle{History of Genetics}
	
\begin{columns}[T]
	\begin{column}{.7\textwidth}
			Humans have probably pondered inheritence for all history:
			\vspace{10pt}
			\begin{itemize}
				\item For much of history, the mechanisms of inheritence were basically unknown
				\item The inheritance of acquired characteristics was widely accepted for much of history (from Hippocrates to Aristotal to Lamarck) \pause
				\item \emph{Early microscopists thought that they had seen small humans inhabiting sperm cells!}
			\end{itemize}
	\end{column}
	\begin{column}{.3\textwidth}
			% Your image included here
% TODO: \usepackage{graphicx} required
\centering
\includegraphics[keepaspectratio, width  = 0.8\textwidth]{img/homunculous}\footnotemark[1]


	\end{column}
\end{columns}
   \note{
	The inheritance of acquired characteristics is often referred to simply as Lamarkism after Jean-Baptiste Lamarck.

	Lamarck was an 19th century evolutionary biologist who formalised a lot of the contemporary thought on how biodiversity originated. The classic example is a giraffe stretching up to reach higher leaves would likely give birth to offspring with longer necks.
	
	\includegraphics[keepaspectratio, width  = 0.8\textwidth]{img/lamarck}\\
			\url{https://simple.wikipedia.org/wiki/Lamarckism}
}
\end{frame}


%%% Slide 7
\begin{frame}
	\frametitle{History of Genetics}
	
	\begin{columns}[T]
		\begin{column}{.6\textwidth}
			By the 19th Century, the dominant theory was \textbf{blending inheritance}
		
			\vspace{5pt}
			\begin{itemize}
				\item The notion that an offspring's traits are simply the average of the parents' traits. 
				\item This is intuitively appealing -  continuously varying traits are often intermediate between their parents 
				\item \textit{There is one big problem with blending inheritance!}
			\end{itemize}
		\end{column}
		\begin{column}{.3\textwidth}
				\centering
			\includegraphics[keepaspectratio, width  = \textwidth]{img/blending}
		\end{column}
	\end{columns}
	

\end{frame}
	%%% Slide 8
\begin{frame}
		\frametitle{The Problem with Blending}
		
				\Huge \centering \emph{What's the big problem with blending inheritance?}
		   \note{
		   	The halving of variation each generation due to blending inheritance was first formalised by Fleeming Jenkin - the inventor of the cable car.\\
		   	Thus, under blending inheritance, half of the variation we see in natural populations at any given time would have to replenished each generation!
		   
		   }
\end{frame}




%%% Slide 9
\begin{frame}
	\frametitle{Darwin's Thoughts on Inheritance}

	\begin{columns}
		\begin{column}{0.4\textwidth}
			\includegraphics[keepaspectratio, width  = \textwidth]{img/iThnk} \pause
		\end{column}
		\begin{column}{0.5\textwidth}
	\small
	\textit{```The laws governing inheritence are quite unknown; no one can say why the peculiarity in different individuals of the same species... is sometimes inherited and sometimes not so"} \footnote[1]{\textit{Ch. 1, The Origin of Species, C. Darwin 1859}} \pause
	
	\vspace{10pt}
	
	But, Darwin clearly appreciated the limitations of blending and felt the need for an alternative:
	
	\vspace{10pt}
	
	\textit{``Each parent transmits it peculiarities, therefore if varieties allowed to cross... such varieties will be constantly demolished" \footnote[2]{\textit{ Foundations of the `Origin of Species', F. Darwin 1909}}}
\end{column}
	\end{columns}
	
	
	
\end{frame}


%%% Slide 9
\begin{frame}

	\frametitle{Trait Variation}

	Blending inheritence only really makes sense when you are thinking about continuously varying traits
	
	\vspace{5pt}

But different modes of variation are common:\pause
\begin{itemize}
	\item Continuous  - traits measured on a numerical scale (e.g. height, diameter, chlorophyll fluorescence) \pause
	\item Discrete - traits that exhibit categorical differences (e.g. different leaf forms, distinct flower colour) \pause
	\item Ordinal - discrete traits with some informative order (e.g. high, medium and low shade tolerance)
	
\end{itemize}

\end{frame}


%%% Slide 10
\begin{frame}
	
	\frametitle{Particulate Inheritance}
	
	\begin{columns}[T]
		
		\begin{column}{0.6\textwidth}
			
			Through careful experimentation analysing discrete traits in peas, Augustinian Friar Gregor Mendel found evidence supporting a model of particulate inheritence
			
			\vspace{10pt}
			
			\includegraphics[keepaspectratio, width  =0.8\textwidth]{img/peas}

		\end{column}
		\begin{column}{.4\textwidth}
			\includegraphics[keepaspectratio, width  =\textwidth]{img/mendel}
			\centering
			Mmmmm...\\
			Peas Peas Peas Peas Peas
		\end{column}
		
		\note{Pea pic from: \url{https://www.thoughtco.com/domestication-history-of-peas-169376}} 
	

		
	\end{columns}
\end{frame}


\begin{frame}
	
	\frametitle{Particulate Inheritance}
\textbf{Particulate Inheritance:} traits are passed from parent to offspring via particles \pause
\vspace{20pt}
\small
\begin{columns}
		\begin{column}{.5\textwidth}
			\underline{Blending Inheritance}
				\begin{itemize}
					\item{Offspring exhibit averages of parental traits}
					\item{The "blended" traits are transmitted to offspring}
					\item{Variation is rapidly lost across generations}
					
				\end{itemize}
			\end{column}
		\begin{column}{.5\textwidth}
			\underline{Particulate Inheritance}
				\begin{itemize}
					\item{Offspring exhibit  \textit{combinations} of parental traits}
					\item{Parental traits can manifest in offspring (or skip generations)}
					\item{Variation is maintained over time}					
				\end{itemize}
			\end{column}
	\end{columns}
	
\end{frame}



%%% Slide 11

\begin{frame}
	\frametitle{Mendel's Crosses}
	\centering
	Mendel examined variation and inheritence of several discrete characteristics of pea plants
	\vspace{10pt}

			\includegraphics[keepaspectratio, width  =0.8\textwidth]{img/mendelCross_2}
\note{Figure from: \url{https://opentextbc.ca/biology/wp-content/uploads/sites/96/2015/02/Figure_08_01_03.jpg}}
\end{frame}
%%% Slide 12

\begin{frame}
	
	\frametitle{Mendel's Crosses}

	
\begin{columns}
	\begin{column}{0.5\textwidth}
			Garden peas are capable of self-fertilization, so Mendel was able to generate "true" lines of peas that exhibited a particular trait/phenotype
		\begin{itemize}
	\item	Crossing lines produces an F1 generation \note{F1 stands for first filial generation. Subsequent crosses of individuals from the F1 generation will produce an F2, crossing individuals from the F2 generation will produce an F3 and so on...}	
	\item 	The patterns of variation among the F2 generations were Mendel's focus
		\end{itemize}
		


	\end{column}
	\begin{column}{0.6\textwidth}  
\begin{center}
			\includegraphics[keepaspectratio, width  =0.6\textwidth]{img/mendelCross_1}
\end{center}
\end{column}
\end{columns}
\end{frame}



\begin{frame}
	
	\frametitle{Mendel's Crosses}
\begin{center}
				\includegraphics[keepaspectratio, width  =0.6\textwidth]{img/crossedPeas}
				\blfootnote{Note the 3:1 ratios of the two pea phenotypes in the F2}
\note{Figure 3 from: van Dijk, P.J., Jessop, A.P. \& Ellis, T.H.N. How did Mendel arrive at his discoveries?. Nat Genet 54, 926–933 (2022). https://doi.org/10.1038/s41588-022-01109-9}
	
	
\end{center}
\end{frame}


\begin{frame}
	
	\frametitle{Mendel's Laws}

The patterns of variation that Mendel observed led him to develop three laws of inheritance
\begin{itemize}
	\item Law of Segregation 
	\item Law of Dominance 
	\item Law of Independent Assortment
		\end{itemize}
	\end{frame}



%%% Slide 14


\begin{frame}
	\frametitle{Mendel's Laws}
	\textbf{The law of segregation: } each individual possesses a pair of particles for any particular trait and each parent passes one of these randomly to its offspring \pause

	\vspace{10pt}

		\textbf{The law of dominance: } for some traits, the presence of one kind of particle masks the presence of another. Mendel referred to the \textbf{dominant} particle as masking the effects of the \textbf{recessive} particle \pause
		
	\vspace{10pt}
		
			\textbf{The law of independant assortment:} when two individuals differ in more than two pairs of traits (e.g. smooth v. wrinkly and green v. yellow), the inheritance of one pair of traits is independent of another 
		
\end{frame}



\begin{frame}
	
	\frametitle{Mendel's Laws}
	\begin{columns}
	\begin{column}{0.6\textwidth}
		\includegraphics[keepaspectratio, width  =\textwidth]{img/crossedPeas}

\end{column}
\begin{column}{0.4\textwidth}
\small	
	\textbf{The law of segregation: } each individual possesses a pair of particles for any particular trait and each parent passes one of these randomly to its offspring
	\vspace{10pt}
		
	How does the image demonstrate \textbf{the law of segregation}?  \pause
	
	\vspace{10pt}
			
	Answer: \textit{Individuals (i.e. seeds) in the F2 generation exhibit a combination of seed colours and textures}
	
	
\end{column}		
		
	\end{columns}
\end{frame}



\begin{frame}
	
	\frametitle{Mendel's Laws}
	\begin{columns}
		\begin{column}{0.6\textwidth}
			\includegraphics[keepaspectratio, width  =\textwidth]{img/crossedPeas}
			
		\end{column}
		\begin{column}{0.4\textwidth}
			\small	
			
			\textbf{The law of dominance: } for some traits, the presence of one kind of particle masks the presence of another.

			\vspace{10pt}
			
			How does the image demonstrate \textbf{the law of dominance}?  \pause
			
			\vspace{10pt}
			
			Answer: \textit{The uniformity of trait values in the F1 generation}
			
			
		\end{column}		
		
	\end{columns}
\end{frame}


\begin{frame}
	
	\frametitle{Mendel's Laws}
	\begin{columns}
		\begin{column}{0.6\textwidth}
			\includegraphics[keepaspectratio, width  =\textwidth]{img/crossedPeas}
			
		\end{column}
		\begin{column}{0.4\textwidth}
			\small	
			\textbf{The law of independant assortment:} when two individuals differ in more than two pairs of traits, the inheritance of one pair of traits is independent of another 
			\vspace{10pt}
			
			How does the image demonstrate \textbf{the law of  independant assortment}? \pause
			
			\vspace{10pt}
			
			Answer: \textit{The fact that wrinkly green peas and smooth yellow peas are seen in the F2 generation}			
			
			
		\end{column}		
		
	\end{columns}
\end{frame}




\begin{frame}
	
	\frametitle{Mendel's Laws}
	\begin{columns}
		\begin{column}{0.6\textwidth}
			\includegraphics[keepaspectratio, width  =\textwidth]{img/crossedPeas}
			
		\end{column}
		\begin{column}{0.4\textwidth}
			\small	
		I count 38 F2 seeds \\
		\bigskip
		13 Green : 25 Yellow\\
		
		9 Wrinkly : 29 Smooth
		
			\vspace{10pt}
		Why do we see these ratios?
			
		\end{column}		
		
	\end{columns}
\end{frame}


\begin{frame}
	\frametitle{Mendelian Terminology}

	
	 Remember, Mendel crossed "true" green (\textbf{G}) peas with "true" yellow (\textbf{Y}) peas.

\bigskip

	The table below gives the results of the self-fertlization of the F1 generation\\
	
	\bigskip	
\begin{columns}	
	\begin{column}{0.43\textwidth}

\begin{flushright}
\begin{tabular}{c|c|c|c|}
	\multicolumn{2}{c|}{} & \multicolumn{2}{c|}{GY} \\
	\multicolumn{2}{c|}{} & G & \textbf{Y} \\
	\hline 
	\multirow{2}{*}{\textbf{YG}}	& G& GG& \textbf{YG }\\		
	\cline{2-4} 
	&	\textbf{Y} & \textbf{YG }& \textbf{YY} \\
	\hline
\end{tabular}

\bigskip

\small \textit{This table is an example of a Punnett square - Yellow phenotypes are shown in \textbf{bold}} \note{Punnett squares are a very useful way of visualising the results of a particular cross. If the possible genotypes can be described, application of Mendel's laws can elucidate the different possible outcomes of a cross.  \\
	\bigskip

For example, here's another Punnett square that shows the results of a cross of a heterozygote with a homozygote:
\bigskip
\begin{tabular}{c|c|c|c|}
	\multicolumn{2}{c|}{} & \multicolumn{2}{c|}{\textbf{GG}} \\
	\multicolumn{2}{c|}{} & \textbf{G} & \textbf{G} \\
	\hline 
	\multirow{2}{*}{\textbf{YG}}	& \textbf{G}& GG& GG \\		
	\cline{2-4} 
	&	\textbf{Y} & YG & GY \\
	\hline
\end{tabular}



 } 
\end{flushright}

\end{column} \pause

	\begin{column}{0.7\textwidth}
\begin{itemize}
\item Yellow is dominant to Green, so any offspring possessing a single Y particle will be yellow
\item So $\frac{3}{4}$, (or 75\% or 0.75 )of the offspring are expected to be yellow
\item With 38 F2 seeds, we would expect 28.5 seeds to be yellow, but \textbf{we would also expect variation around this number}

\end{itemize}
	\end{column}
	\end{columns}
	
	
\end{frame}



\begin{frame}
	\frametitle{Mendelian Terminology}
	
	
	Remember, Mendel crossed "true" green (\textbf{G}) peas with "true" yellow (\textbf{Y}) peas.
	
	\bigskip
	
	The table below gives the results of the self-fertlization of the F1 generation\\
	
	\bigskip	
	\begin{columns}	
		\begin{column}{0.4\textwidth}
			
			\begin{flushright}
				\begin{tabular}{c|c|c|c|}
					\multicolumn{2}{c|}{} & \multicolumn{2}{c|}{\textbf{GY}} \\
					\multicolumn{2}{c|}{} & \textbf{G} & \textbf{Y} \\
					\hline 
					\multirow{2}{*}{\textbf{YG}}	& \textbf{G}& GG& YG \\		
					\cline{2-4} 
					&	\textbf{Y} & YG & YY \\
					\hline
				\end{tabular}
			\end{flushright}
		\end{column} 
		
		\begin{column}{0.7\textwidth}
			\begin{itemize}
			\item We use the term \textbf{homozygote} to refer to the offspring possessing the GG or the YY combinations
			\item We use the term \textbf{heterozygote} to refer to both YG and GY offspring as they are equivalent\blfootnote{but not always - see notes}
			\end{itemize}
		\end{column}
	\end{columns}
	\note{ In some cases, the parent of origin for a particular allele will dertermine aspects of how that allele is expressed. Remember this when we get to the section on epigenetics...}
	
\end{frame}









\begin{frame}
	\frametitle{Codominance}
	
	\begin{center}
		\newcommand{\picC}{\includegraphics[ height=1in]{img/greenSpruceCone}}
		\newcommand{\picD}{\includegraphics[ height=1in]{img/redSpruceCone}}
		\newcommand{\picE}{\includegraphics[ height=1in]{img/reddishSpruceCone}}
\small
Codominance is a form of inheritance wherein both alleles in a heterozygote are equally expressed \\

As seen in this example from Sitka Spruce



				\parbox{\widthof{\picC}}{\picC} $\times$ 
		\parbox{\widthof{\picD}}{\picD}
		$\,\to\,$  \pause
		\parbox{\widthof{\picE}}{\picE}  \\
\bigskip
This was not discovered by Mendel, however

\pause
		\bigskip
		\parbox{\widthof{\picC}}{\picC} $\times$ 
		\parbox{\widthof{\picD}}{\picC}
		$\,\to\,$  \pause
				\parbox{\widthof{\picE}}{\picC} 
	\end{center}	
	\normalsize
\note{Codominance is superficially consistent with blending inheritence. Many traits that are not under the control of individual genes may exhibit patterns like codominance, but this is also consistent with predictions from the infinitesimal model (see later on in the lecture).}
	
\end{frame}



\begin{frame}
	\frametitle{Codominance}

Codominant and full dominance are just two domains on a continuous range

\bigskip

The degree of dominance can vary arbitrarily


		\begin{block}

		\only<1>{
			\centering				\includegraphics[keepaspectratio,width=3.5in]{img/codominantPlot} 
		} 
		\only<2>{
			\centering				
			\includegraphics[keepaspectratio,width=3.5in]{img/domCodomPlot} 
			}  
		\only<3>{
	\centering				
	\includegraphics[keepaspectratio,width=3.5in]{img/allDom} 
}  

	\end{block}


\end{frame}


%%%%%%%%

\begin{frame}
	\frametitle{Crossing Experiments}
	Leaf phenotypes in European beech, \textit{Fagus sylvatica}
	\begin{center}
	\newcommand{\picA}{\includegraphics[keepaspectratio, height=2.25in,width=1.6in]{img/roundBeechLeaf}}
	\newcommand{\picB}{\includegraphics[keepaspectratio, height=2.25in,width=1.6in]{img/cutBeechLeaf}}
	\Huge
	\parbox{\widthof{\picA}}{\picA} $\times$ 
	\parbox{\widthof{\picB}}{\picB} 
\end{center}	
	\normalsize
	A leaf shape trait controlled by a single gene 
	\bigskip
	
	Assuming the two individuals are homozygotes, how could you figure out if the allele for the cut leaf phenotype is dominant, recessive or codominant?
	
\end{frame}


\begin{frame}
	\frametitle{Test Crosses}
	Test crosses are used to determine individual genotypes
	\bigskip 
	
	In a test cross,  individuals with unknown genotype (WW or Ww?) are crossed with individuals homozygous for a recessive trait (ww)

	\begin{center}
		\newcommand{\picA}{\includegraphics[keepaspectratio, height=2.25in,width=1.6in]{img/roundBeechLeaf}}
		\newcommand{\picB}{\includegraphics[keepaspectratio, height=2.25in,width=1.6in]{img/cutBeechLeaf}}
		\Huge
		\parbox{\widthof{\picA}}{\picA} $\times$ 
		\parbox{\widthof{\picB}}{\picB} 
	\end{center}	

	If any of the offspring exhibit the recessive phenotype, the unknown parent must be... ?\pause \textbf{ Heterozygous}
	
	
\end{frame}


\begin{frame}

\Huge
Questions? \\ \pause
Let's take a short break
	
\end{frame}



%%% Slide 13

\begin{frame}
	
	\frametitle{Particulate Inheritance and Classical Genetics}
	
	\begin{columns}[T]
		
		\begin{column}{0.6\textwidth}
			
			\begin{itemize}
				\item Proposed in 1865 and 1866
				\item 6-7 years after Darwin’s Theory of Evolution
				\item As far as anyone knows, Darwin was totally unaware of Mendel\textsuperscript{\textit{but see notes!}}
				\vspace{20pt}
				\item Represents the foundation of classical genetics
				\item \textbf{Classical genetics} refers to the study of genetic patterns observable from reproductive events
			\end{itemize}
			
			
		\end{column}
		\begin{column}{.4\textwidth}
			\includegraphics[keepaspectratio, width  =\textwidth]{img/mendel}
			\centering
			More peas please
		\end{column}
		
		\note{Pea pic from: \url{https://www.thoughtco.com/domestication-history-of-peas-169376}}
		
		
	\end{columns}
	\note{
		A letter from Darwin to Alfred Russel Wallace - 6th Februrary 1866 \\
		
		``My dear Wallace,
		
		After I had despatched my last note, the simple explanation which you give had occurred to me, and seems satisfactory.
		
		I do not think you understand what I mean by the non-blending of certain varieties. It does not refer to fertility; an instance will explain; I crossed the Painted Lady and Purple sweet-peas, which are very differently coloured vars, and got, even out of the same pod, both varieties perfect but none intermediate. Something of this kind I shd. think must occur at first with your butterflies got the 3 forms of Lythrum; tho’ these cases are in appearance so wonderful, I do not know that they are really more so than every female in the world producing distinct male got female offspring.
		
		I am heartily glad that you mean to go on preparing your journal.6
		
		Believe me yours,  very sincerely,
		
		Ch. Darwin"}
		
\end{frame}



\begin{frame}
	
	\frametitle{Towards a Synthesis}
	
	Mendel's contributions were underappreciated in his time \\
	
	\bigskip
	Mendel's findings began to be appreciated early in the 1900s, largely thanks to the work of William Bateson and Edith Rebecca Saunders\\
	

	\bigskip

	

\note{	``\textit{This purity of the germ-cells, and their inability to transmit both of the antagonistic characters, is the central fact proved by Mendel's work. We thus reach the conception of unit-characters existing in antagonistic pairs. Such characters we propose to call allelomorphs, and the zygote formed by the union of a pair of opposite allelomorphic gametes, we shall call a heterozygote. Similarly, the zygote formed by the union of gametes having similar allelomorphs, may be spoken of as a homozygote}" \\
	This paper is where we get the terms \textbf{heterozygote, homozygote} and \textbf{allele} (which was shortened from the more cumbersome \textit{allelomorph})
	} 


\end{frame}
	
	

\begin{frame}
	
	\frametitle{Towards a Synthesis}
\begin{center}
The rediscovery of Mendel's laws kicked off a scientific feud \\\textbf{The Biometricians v. The Mendelians}
\end{center}
\begin{columns}
\begin{column}{0.5\textwidth}
	\begin{flushright}
	Studying continuous variation
	\end{flushright}
	\end{column}
	\begin{column}{0.5\textwidth}
	\begin{flushleft}
		Studying discrete variation
	\end{flushleft}
\end{column}

\end{columns}

\vspace{3pt}

\hrule \pause
\vspace{5pt}

The debate boiled down to the following question:
\vspace{5pt}

\begin{center}
	\textit{How can the inheritence of discrete particles explain patterns of continuous variation?}
\end{center}
\blfootnote{ \textit{Historians suggest that much of the debate was driven by personality rather than intellectual difference though} }
 

\note{ 
	Biometricians, typically concerned with patterns of continuous variation, did not accept that Mendelian inheritence could explain quantitative traits 
	
	The Mendelians, convinced by the clarity of Mendel's and subsequent experiments (e.g. those of Bateseon and Saunders), accepted Mendelian inheritance as an explanation for trait variation - though the logic they used to reconcile the arguments of the biometricians are not clear. 
	
	The Mendelian Biometrician debate was rather fierce. ``Mr. Bateson devotes many words to these questions, but one cannot help feeling that his speculations would have had more value had he kept his emotions under better control; the style and method of the religious revivalist are ill-suited to scientific controversy. It is difficult to speak with patience either of the turgid and bombastic preface to 'Mendel’s Principles,' with its reference to Scribes and Pharisees, and its Carlylean inversions of sentence, or of the grossly and gratuitously offensive reply to Professor Weldon and the almost equally offensive adulation of Mr. Galton and Professor Pearson"  Yule 1990}



\end{frame}



%%% Slide 14


\begin{frame}
\frametitle{Reconciling the Mendelians and the Biometricians}

\begin{columns}
	\begin{column}{0.4\textwidth}
\small	Imagine a species of tree with a codominant trait we care about (let's say tree height) that is not affected by the environment\note{What a silly assumption to make though - plant height would of course be influenced by myriad factors beyond genetics alone (climatic conditions, substrate, competition etc.)}\\\pause

\bigskip

\small In this species there is a single gene that controls height - with a pair of alleles (just like yellow v. smooth peas). The $A$ allele does not affect height, but the $a$ allele leads to a height of $+1$ \\ \pause

\bigskip



\end{column}
\begin{column}{0.6\textwidth}
Pretend that we conducted a crossing experiment on these plants just like Mendel\\
 \textit{What possible trait values could the F2 generation exhibit?} \\
\pause
\bigskip

\includegraphics[keepaspectratio, width = \textwidth]{img/oneGene} \\  
\end{column}
\end{columns}	



\end{frame}


\begin{frame}
	\frametitle{Reconciling the Mendelians and the Biometricians}
	
	\begin{columns}
		\begin{column}{0.4\textwidth}
			\small	Now let's say that our tree height trait is controlled by 100 genes, inherited according to the law of independant assortment\\
			
			\bigskip
			
			\small Instead of a $+1$ effect on height, each one has an effect of $+\frac{1}{100}$
			 
						\bigskip
			\textit{What would the distribution of possible trait values look like now?} 
			
		\end{column}
		\begin{column}{0.6\textwidth}
			
			\includegraphics[keepaspectratio, width = \textwidth]{img/oneGene} \pause
			
			\includegraphics[keepaspectratio, width = \textwidth]{img/oneHundredGenes}

		\end{column}
	\end{columns}	
	
	
	
\end{frame}




\begin{frame}
	\frametitle{Reconciling the Mendelians and the Biometricians}

Assuming Mendel's laws and that a large number of genes controls complex traits can reconcile the Mendelian and Biometrician's arguments \\

\bigskip

This model was first formalized in 1918\footnote[1]{This idea, represents a major turning point in the history of genetics}, but even Mendel had an inkling that this may be the case \\

\bigskip

If we take this model to the limit of an infinite number of genes, each making an infinitely small contribution to a particular trait we get \textbf{the infinitesimal model} - the basis of quantitative genetics (more on that in Module 3) 

\end{frame}

\begin{frame}
	
	\frametitle{How can we Apply a Model of Genetics?}
	
	
\begin{columns}
	\begin{column}{0.3\textwidth}	
		Quantitative genetic models work and have been very effective in the last 100 years!
	\end{column}
	\begin{column}{0.7\textwidth}
	\centering
	\includegraphics[keepaspectratio, width  =0.8\textwidth]{img/zuidhof_2014} \footnote {Modified from Figure 1 - Zuidhof et al. 2014}
\end{column}
\end{columns}
	
\end{frame}


\begin{frame}
	
	\Huge
	Questions? \\ \pause
	Let's take a short break
	
\end{frame}




\begin{frame}

	\frametitle{The Units of Heredity}
	
	So a particulate theory of inheritance can explain discrete and continuous traits\\
	\bigskip
	
	\textbf{But what are the particles??} \pause

\tiny It should be obvious that we are talking about chromosomes and DNA	
	
	
	
\end{frame}


\begin{frame}
	\frametitle{A Timeline of Some Discoveries}
	
	\begin{columns}
\begin{column}{0.4\textwidth}
	\includegraphics[keepaspectratio, width  =\textwidth]{img/chromosomes}
\end{column}
	\begin{column}{0.6\textwidth}
		\begin{itemize}
			\small
		\item[1865]  Mendel postulates laws of inheritance \pause
		\item[1869]  DNA Isolated - though it was unclear what its relevence was unclear \pause
		\item[1882] Discovery of the fibrous network of "chromatin" (\textit{stainable material}) and chromosomes within nuclei \pause
		\item[1902-6]  Sutton-Boveri chromosome theory - the segregation of chromosomes during meiosis matches the segregation pattern of Mendel’s laws \pause
		\item[1915] Morgan demonstrated that chromosomes carry genes, and also discovered genetic linkage\textsuperscript{\textit{won Nobel Prize in 1933}}

	\end{itemize}
	\end{column}

	\end{columns}
		\blfootnote{Drawing of mitosis by Walther Flemming 1882}

	
\end{frame}

\begin{frame}
	
\frametitle{Chromosomes}

\begin{columns}
	
\begin{column}{0.4\textwidth}
	\begin{itemize}
	\item A chromosome is a string-like structure holding the genetic material of an organism
	\item  The figure shows a micrograph of fluorescently stained loblolly pine chromosomes
\end{itemize}
\end{column}
\begin{column}{0.6\textwidth}
			\includegraphics[keepaspectratio, width  =\textwidth]{img/loblollyChroms}
\end{column}
\end{columns}
\blfootnote{We will only cover the inheritance of chromosomes here and leave detailed molecular features about DNA and chromosomes to next week}
\end{frame}

\begin{frame}
	\frametitle{Conifer Life Cycle}
	\begin{columns}
	\begin{column}{0.8\textwidth}
	\includegraphics[keepaspectratio, width  =\textwidth]{img/coniferLifeCycle}
	\end{column}
	\begin{column}{0.3\textwidth}
	\small 	 The specific stages of the lifecycles vary across taxa, the take home message here is that meiosis produces \textit{haploid} gametes	 \blfootnote{ Modified from Neale et al 2014 BMC Genomics}
	\end{column}
	\end{columns}

\end{frame}


\begin{frame}
	\frametitle{Homologous chromosomes}

\begin{columns}

	\begin{column}{0.5\textwidth}
\begin{itemize}
	\item[] Chromosomes are in pairs except in gametes (pollen and ovules)

	\item Each pair of chromosomes are called homologous chromosomes, one is carried in the pollen, the other in the ovule
	\item Each member of a pair carries the same genes (except allelic variation)
\end{itemize}
\end{column}
	\begin{column}{0.5\textwidth}
	\includegraphics[keepaspectratio, width  =\textwidth]{img/homologousChroms}
	\vspace{10pt}\\
	
		\tiny	\textbf{Karyotype} (chromosome configuration) of Maize	\blfootnote{Modified from Mondin et al 2014 Front. Plant Sci.}
	\includegraphics[keepaspectratio, width  =\textwidth]{img/maizeKaryotype}


\end{column}
\end{columns}

\end{frame}



\begin{frame}
	\frametitle{Confirming Mendel's First Law}
	
	\begin{columns}
		
		\begin{column}{0.5\textwidth}
			\begin{itemize}

\small 
\item[]	\textbf{The law of segregation: } each individual possesses a pair of \st{particles}  homologous chromosomes and each parent passes one of these randomly to its offspring\\
\vspace{20pt}

\item[] The First Law helped to figure out the genetic features of chromosomes, while the discovery of chromosomes and how they are transmitted confirms the First Law 
				
			\end{itemize}
		\end{column}
		\begin{column}{0.5\textwidth}
			\includegraphics[keepaspectratio, width  =\textwidth]{img/homologousChroms}
			\vspace{10pt}\\
			
			\tiny	\textbf{Karyotype} (chromosome configuration) of Maize	\blfootnote{Modified from Mondin et al 2014 Front. Plant Sci.}
			\includegraphics[keepaspectratio, width  =\textwidth]{img/maizeKaryotype}
			
			
		\end{column}
	\end{columns}
	
\end{frame}




\begin{frame}
	\frametitle{Confirming Mendel's Second Law}
	
	\begin{columns}
		
		\begin{column}{0.5\textwidth}
			\begin{itemize}
				
				\small 
				\item[]	\textbf{The law of independant assortment:} traits are inherited independently of one another \
				\vspace{10pt}
				
				\item[] During meiosis, different chromosome are replicated and segregate independently
								\vspace{10pt}
				\item[] If a pair of genes are located on different chromosomes, they obey the Second Law
				
			\end{itemize}
		\end{column}
		\begin{column}{0.5\textwidth}
			\includegraphics[keepaspectratio, width  =\textwidth]{img/independentAssortment}

					
			
		\end{column}
	\end{columns}
	
\end{frame}




\begin{frame}

\begin{center}
\frametitle{Chromosome Combinations}
			\includegraphics[keepaspectratio, width  =0.3\textwidth]{img/loblollyChroms}\\
			
What is the probability that crossing two loblolly pines will result in two genetically identical offspring?
\end{center}
\begin{itemize}
	\item 2 pairs of chromosomes  $= 4$ combinations 
	\item 12 pairs in loblolly pine, How many combinations?\pause 
	\item $= 2^{12} = 4,096$ combinations
	\item So, very small chance for a particular cross to lead to 2 genetically identical offspring 
\end{itemize}


\end{frame}

%%% Slide 13


%%% Slide 3
\begin{frame}	

	\Large \textbf{Genetics is the study of genes}, of variation and heredity across all branches of the tree of life	
	
	
\end{frame}


\begin{frame}
	
	\frametitle{Branches of Genetics}
	
	\begin{columns}[T]
		
		\begin{column}{0.45\textwidth}
			\begin{itemize}
			\item{Behavioural genetics}
			\item{\textbf{Classical genetics}}
			\item{Developmental genetics}
			\item{\textbf{Conservation genetics}}
			\item{\textbf{Ecological genetics}}
			\item{\textbf{Evolutionary genetics}}
			\item{\textbf{Genecology}}
			\item{Genetic engineering}
			\item{Genomics}
			
			\end{itemize}
			
		\end{column}
		\begin{column}{.45\textwidth}
			\begin{itemize}
			\item{Medical genetics}
				\item{Forensics}
				\item{Molecular genetics}
				\item{\textbf{Quantitative genetics}}
				\item{\textbf{Population genetics}}
				\item{Phylogenetics}
				\item{Statistical genetics}
				\item{Genetic epidemiology}
				\item{Archaeogenetics}
			\end{itemize}
		\end{column}
		\end{columns}
		\blfootnote{A non-complete and \textit{slightly} biased list}
\end{frame}




\begin{frame}

\frametitle{Terminology Check}

We'll expand on each of these throughout the course...\\

\begin{itemize}
	\item Gene - the unit of heredity
	\item Allele - a genetic variant 
	\item Genotype - the configuration of genes possessed by an individual
	\item Homozygote - an individual possessing two identical alleles
	\item Heterozygote - an individual possessing two distinct alleles
	\item Any others... ?
	
\end{itemize}

\end{frame}

\begin{frame}
	
	\frametitle{Learning Outcomes}
	
	\Large
	\begin{itemize}
	\item   Basic definitions in genetics
	\item	Basic principles and terms in classical genetics
	\item   How Mendelian inheritence can lead to continuous trait variation 
	\item	Molecular mechanisms of Mendelian inheritence
	\end{itemize}
	
\end{frame}

%%% Slide 15

\begin{frame}
	\frametitle{A Question to Think About...}
	\begin{center}
		\includegraphics[keepaspectratio,width=3.5in]{img/albino} \\
		Albinism is caused by a recessive lethal allele\\
		Seedlings with albinism have no chlorophyll, thus no photosynthesis\\
		How come tree populations still carry this allele? 
	\end{center}
\end{frame}


\begin{frame}
	\frametitle{Assigned Reading}
	\centering
			\fbox{\includegraphics[keepaspectratio,width=3in]{img/AGF_paper}}\\
	
\begin{itemize}
	\item 	Take the time you would spend in class next Tuesday to read this paper in depth
	\item Make a list of the terms and concepts you do not understand\\
	\item The contents of this paper istabular potentially examinable
	
	\bigskip
	\item 	I will upload the PDF to Canvas after class

\end{itemize}
	
\end{frame}


\begin{frame}[fragile]
	\frametitle{Extra Material}
	
	\textbf{Below is the R code to make the figures on the infinitesimal model - feel free to play around with it}
	\begin{adjustbox}{max width=\textwidth}
		
	\begin{lstlisting}[language=R]

# Demonstrate the distribution of trait values for a quantitative trait
# Under Mendelian segregation for an arbitrary number of genes
# Assumes random mating, constant effect sizes, constant allele frequencies
	
nGenes = 100
alleleFrequency = 0.2
popSize = 5000
effectSize = 1
			
hist( 
	replicate(popSize,
		sum(  1 * rbinom(nGenes, 2, alleleFrequency) ) ),
	col = "#e69b99",
	xlab=  "Trait Value",
	main= paste("Distribution of Trait Values in F2s Assuming\n",nGenes, 
		"Genes Segregating According to Mendelian Inheritance"),
	breaks = 40)
		\end{lstlisting}
\end{adjustbox}
\end{frame}


%%% Slide 17
\end{document}




